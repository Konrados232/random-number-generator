\documentclass[60pt]{article}
\usepackage[a4paper, margin={1in, 1in}]{geometry}
\usepackage[utf8]{inputenc}
\usepackage{polski}
\usepackage{mathtools}
\usepackage{amsfonts}
\usepackage{amssymb}
\usepackage{amsmath}
\usepackage{multicol}
\usepackage{paralist}
\usepackage{tabto}
\usepackage{graphicx}
\usepackage{etoolbox}
\usepackage{changepage}
\usepackage{tasks}
\usepackage{pgfplots}
\usepackage{fancyhdr}
\usepackage{mathtools}
\usepackage{tikz}
\DeclarePairedDelimiter\ceil{\lceil}{\rceil}
\DeclarePairedDelimiter\floor{\lfloor}{\rfloor}
\usepackage{graphicx}

\usepackage{minted}
\usemintedstyle{autumn}

\DeclareMathOperator{\arctg}{arctg}
\DeclareMathOperator{\sh}{sh}
\DeclareMathOperator{\ch}{ch}
\DeclareMathOperator{\sgn}{sgn}

\let\arctan\relax
\DeclareMathOperator{\arctan}{arctg}
\let\tan\relax
\DeclareMathOperator{\tan}{tg}

\pagestyle{fancy}

\title{Generator liczb pseudolosowaych}
\author{Konrad Nowak}

\begin{document}

\maketitle

\newpage
\tableofcontents

\newpage

\section{Wstęp}
\subsection{Cel projektu}
Głównym celem projektu jest zaimplementowanie standardowego generatora liczb pseudolosowych bez korzystania z wbudowanych bibliotek losujących liczby. Z głównego generatora będą korzystały generatory m.in. różnych rozkładów w tym:
\begin{itemize}
    \item Generator liczb zmiennoprzecinkowych w przedziale $(0,1)$
    \item Generator Bernoulliego
    \item Generator rozkładu dwumianowego
    \item Generator rozkładu Poissona
    \item Generator rozkładu wykładniczego
    \item Generator rozkładu normalnego
\end{itemize}
Kolejnym z celów jest analiza pseudolosowości oraz poprawności każdego z generatorów. Jakie możliwe poprawki można by było nanieść w implementacji i sprawdzenie, czy wygenerowany ciąg liczb sprawia wrażenie losowego.

\subsection{Implementacja}
Implementacja zawiera 3 pliki:
\begin{itemize}
    \item $Generator.py$ - klasa zawierająca implementację wszystkich generatorów, w konstruktorze przyjmująca wartość ziarna. Dodatkowo posiada metody zmiany i resetu ziarna.
    \item $Test.py$ - klasa zawierająca metody pozwalające testować generatory za pomocą testów serii oraz pozwalająca wypisać histogram generatora podanego w konstruktorze.
    \item $main.py$ - główny plik do kompilowania, w którym są zapisane przykładowe wywołania generatorów i testów.
\end{itemize}

\newpage

\section{Rodzaje generatorów}
\subsection{Generator G}
Generator addytywny postaci:
$$X_n = (a \cdot X_{n-1} + c) \mod m$$
Gdzie:
\begin{itemize}
    \item $X_n$ - nasza wygenerowana liczba
    \item $X_{n-1}$ - poprzednio wygenerowana liczba lub początkowa
    \item $a$ - stała, mnożnik
    \item $c$ - stała, przyrost, spełniająca warunek $NWD(c,m) = 1$
    \item $m$ - moduł
\end{itemize}
W naszym przypadku:
$$X_n = (5^{10} \cdot X_{n-1} + 7151) \mod 2^{31} - 1$$
Generator może wylosować liczby z przedział do maksymalnej wartości $int$, co pozwala na znaczne zróżnicowanie wyników. Stała $c$ oraz $m$ muszą być liczbami względnie pierwszymi, by nie doszło do sytuacji, gdzie pojawia się ciąg zer.
\\
Generator będzie używany jako baza dla reszty przedstawionych generatorów.


\subsection{Generator J}
Generator liczb zmiennoprzecinkowych w przedziale $(0,1)$ przy wykorzystaniu generatora G. Jego działanie jest bardzo proste - biorąc wynik z generatora G dzielimy go przez 
Wykorzystując generator G, otrzymujemy wynik przez wzór $\frac{G() + 1}{m + 1}$, co zagwarantuje, że wyniki będą w przedziale $(0,1)$.

\subsection{Generator B}
Generator dwupunktowy, zwracający $1$ w przypadku sukcesu, bądź $0$ w przypadku porażki. Sprawdzamy wynik z wykorzystanego generatora J, czy wynik jest poniżej lub równy podanego w argumencie $p$ (prawdopodobieństwa), co uznajemy za sukces. W przeciwnym wypadku - porażkę. 

\subsection{Generator D}
Generator rozkładu dwumianowego dla podanych w argumencie $p$ (prawdopodobieństwa) oraz $n$ (ilości prób). Dla każdego coraz to kolejnych wyników z generatora B zliczamy łączną liczbą sukcesów.

\subsection{Generator P}
Jako argument przyjmuje $\lambda$ (wartość oczekiwaną). Generator korzysta z generatora J, mnoży kolejne prawdopodobieństwa wydarzenia się danej liczby zdarzeń w jednej jednostce czasu aż wartość nie przekroczy $q = e^{-\lambda}$ (wylosowana na początku wartość). Rozkład Poissona jest szczególnym przypadkiem rozkładu dwumianowego, w którym nasze zdarzenia rozpatrujemy w coraz to mniejszych przedziałach czasowych, stąd $q = e^{-\lambda}$, gdyż $\lim_{n \to \infty} (\frac{\lambda}{n})^k \cdot(1 - \frac{\lambda}{n})^{n-k} = e^{-\lambda}$.

\subsection{Generator W}
Jako argument przyjmuje $\lambda$ (wartość oczekiwaną). Generator korzysta z obliczonej dystrybuanty dla $f(x) = \lambda e ^ {-\lambda x}$  $F(x) = 1 - e^{-\lambda x}$ i jej odwrotności $F^{-1}(x) = \frac{-\log(U)}{\lambda}$, gdzie $U$ to liczba pomiędzy 0 a 1 z naszego generatora J (we wzorze jest $\log(1 - U)$, ale skoro $U \in (0,1)$, to możemy to po prostu zapisać jako $U$. Rozkład wykładniczy opisuje odstępy czasu między kolejnymi wydarzeniami.

\subsection{Generator N}
Korzysta z metody Boxa-Mullera, która wyznacza parę dwóch zmiennych $(x,y)$ wyznaczonych wzorami $x = \sqrt{-2 \log (1 - U_1)} \cdot \cos(2 \p U_2)i$ oraz $y = \sqrt{-2 \log (1 - U_1)} \cdot \sin(2 \p U_2)$, gdzie $U_1$ i $U_2$ to wartości wygenerowane przez generator J. Następnie przeskalowuje wyliczony $x$ o podaną w argumencie zmienną $o$ oraz przesuniecie $u$ (zamiana standardowego rozkładu na parametryzowany).


\section{Testy}
\subsection{Eksperymenty}
Dla każdego z generatorów zostały przeprowadzone cztery serie eksperymentów: dla ziaren: $454565756$, $246457877$, $897654345$ oraz $986168$. Dla każdego z generatorów zostały przeprowadzone takie same wywołania generatorów (z odpowiednimi zmiennymi wejściowymi):
\begin{minted}{python}
main_test.G_histogram(10000)
main_test.J_histogram(10000)
main_test.B_histogram(10000, 0.75)
main_test.B_histogram(10000, 0.15)
main_test.D_histogram(10000, 0.80, 30)
main_test.D_histogram(10000, 0.20, 100)
main_test.P_histogram(10000, 1)
main_test.P_histogram(10000, 5)
main_test.P_histogram(10000, 10)
main_test.W_histogram(10000, 10)
main_test.N_histogram(10000, 9765625, 7151)
\end{minted}


\subsection{Testy serii}
Testy serii zostały wykonane na takich samych ziarnach na generatorze G oraz J (powinny dawać dokładnie takie same wyniki) i próbkach $10000$ elementów. Test serii polega na wyliczeniu mediany wygenerowanych elementów i podzielenie ich na dwa zbiory $A$ i $B$ - wartości poniżej mediany i powyżej mediany (wartości równe medianie pomijamy). Tworzymy dodatkową tablicę i sprawdzamy, ile jest serii powtarzających się elementów z jednego i drugiego zbioru. Liczba serii będzie podlegała rozkładowi naturalnemu, więc obliczamy wartość średnią i wariancję, a z nich statystykę $Z$. Sprawdzamy, czy $Z$ mieści się między $-1.96$ a $1.96$ ($2,5\%$ z obu stron rozkładu odcinamy, by sprawdzić, czy wynik mieści się w $95\%$ wyników).

\section{Wnioski}
Wynikowe histogramy oraz testy serii pokazały, że generator G jest w stanie generować liczby w miarę pseudolosowe i poprawne z nich rozkłady. Jednakże nie dla każdego ziarna jest to tak dokładne. Do poprawy mogłyby być na pewno współczynniki w generatorze G dobrane znacznie lepiej oraz jego znaczna modyfikacja zwiększająca poczucie losowości danych. 


\newpage

\section{Źródła}
\begin{itemize}
    \item $http://home.agh.edu.pl/~chwiej/mn/generatory_16.pdf$
    \item $https://pl.wikipedia.org/wiki/Test_serii$
    \item $https://eduinf.waw.pl/inf/alg/001_search/0020.php$
    \item $https://pl.wikipedia.org/wiki/Rozkład_Poissona$
    \item $Wieczorkowski_R._-_Komputerowe_generatory_liczb_losowych$
\end{itemize}


\end{document}